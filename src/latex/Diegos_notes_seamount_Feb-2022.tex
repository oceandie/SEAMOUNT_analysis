\documentclass[authoryear]{elsarticle}

\usepackage{lineno,hyperref}
\usepackage{natbib}
\usepackage{amsmath,subcaption}    %% added by MJB to allow split and subfigure
\usepackage{enumerate} %% added by MJB
\usepackage{listings}
\usepackage{array}
\usepackage{booktabs}
\usepackage{fancyvrb}
\usepackage{graphicx}
\graphicspath{ {./Current_Code_Review_Images/} }
\usepackage{bm}
\usepackage{tikz}
\lstset{
basicstyle=\small\ttfamily,
columns=flexible,
breaklines=true
}
\modulolinenumbers[1]
\renewcommand{\thefootnote}{\fnsymbol{footnote}}
%\journal{Journal of Templates}

%%%%%%%%%%%%%%%%%%%%%%%
%% Elsevier bibliography styles
%%%%%%%%%%%%%%%%%%%%%%%
%% To change the style, put a % in front of the second line of the current style and
%% remove the % from the second line of the style you would like to use.
%%%%%%%%%%%%%%%%%%%%%%%

%% Numbered
%\bibliographystyle{model1-num-names}

%% Numbered without titles
%\bibliographystyle{model1a-num-names}

%% Harvard
%\bibliographystyle{model2-names}
%\biboptions{authoryear}

%% Vancouver numbered
%\usepackage{numcompress}\bibliographystyle{model3-num-names}

%% Vancouver name/year
%\usepackage{numcompress}\bibliographystyle{model4-names}\biboptions{authoryear}

%% APA style
%\bibliographystyle{model5-names}\biboptions{authoryear}

%% AMA style
%\usepackage{numcompress}\bibliographystyle{model6-num-names}

%% `Elsevier LaTeX' style
%\bibliographystyle{elsarticle-num}
%%%%%%%%%%%%%%%%%%%%%%%

\begin{document}

\begin{frontmatter}

\title{Development and validation of \\ the SEAMOUNT test case}

\author{Diego Bruciaferri}
\address{Met Office, Fitzroy Rd, Exeter, UK }

\begin{abstract}
\end{abstract}

%\begin{keyword}
%\end{keyword}
\end{frontmatter}

% \linenumbers

\today{}

\section{Introduction} 

We decided to modify the SEAMOUNT test case previously implemented by Amy to include the following enhancements:

\begin{itemize}
	\item[1.] Remove the subtraction of a reference density profile prior the computation of the pressure gradient force, i.e. using NEMO standard Primitive Equations. This should allow us to test the new HPG schemes in a more realistic setup.
	\item[2.] Point 1 allow us to choose a more modern and `similar-to-NEMO' reference publication for the SEAMOUNT test case. After some trials, the study of \cite{Ezer2002} has been chosen for the following reasons:
	\begin{itemize}
		\item  
    \end{itemize}
    \item[3.] stretched coordinates 
    \item[4.] VQS  
\end{itemize}

Section \ref{Sec_first_calcn} lays out one set of calculations. A simpler approach is described in section \ref{Sec_second_calcn}. 
Section \ref{Sec_third_calcn} generalises the calculations to grids whose spacing is smoothly varying. A simpler solution
to that problem is derived in section \ref{Sec_fourth_calcn}  

\section{Test case configuration} \label{config}

\section{Numerical experiments setup} \label{exp_setup}
\begin{table}[htp]
	\centering
	\hspace{-7cm}
	%\begin{center}
	\begin{scriptsize}
		\begin{tabular}{llcccc}
			\cmidrule[0.5pt]{1-6}
			\textsc{\textbf{EXP}} & \textsc{\textbf{Seamount}} & \textsc{\textbf{HPG}} & \textsc{\textbf{EOS}} & \textsc{\textbf{Vertical coord.}} & \textsc{\textbf{rmax}} \\
			\cmidrule[0.5pt]{1-6}
			N-SCO-m      & moderately steep & SCO & EOS-80  & $\sigma$-coord. & 0.07 \\
			N-SCO-s      & very steep       & SCO & EOS-80  & $\sigma$-coord. & 0.36 \\
			N-PRJ-m      & moderately steep & PRJ & EOS-80  & $\sigma$-coord. & 0.07 \\
			N-PRJ-s      & very steep       & PRJ & EOS-80  & $\sigma$-coord. & 0.36 \\
			N-DJC-CTR-m  & moderately steep & DJC & EOS-80  & $\sigma$-coord. & 0.07 \\
			N-DJC-CTR-s  & very steep       & DJC & EOS-80  & $\sigma$-coord. & 0.36 \\
			N-DJC-TEOS-m & moderately steep & DJC & TEOS-10 & $\sigma$-coord. & 0.07 \\
			N-DJC-TEOS-s & very steep       & DJC & TEOS-10 & $\sigma$-coord. & 0.36 \\
			N-DJC-SH94-m & moderately steep & DJC & EOS-80  & $s$-coord.      & 0.07 \\
			             &                  &     &         & \cite{Song1994}  &      \\
			N-DJC-SH94-s & very steep       & DJC & EOS-80  & $s$-coord.      & 0.36 \\
			             &                  &     &         & \cite{Song1994}  &      \\	
			N-DJC-VQS-m  & moderately steep & DJC & EOS-80  & $\sigma$-coord. & 0.05 \\
			N-DJC-VQS-s  & very steep       & DJC & EOS-80  & $\sigma$-coord. & 0.25 \\              
			\cmidrule[.5pt]{1-6}
		\end{tabular}
	\end{scriptsize}
	\hspace{-7cm}
	\caption{ }
\end{table}

\section{Results} \label{results} 

%\begin{figure}[htp]
%	\centering
%	\hspace{-7cm}
%	\begin{tabular}{c}
%		\includegraphics[width = 13cm]{figures/area2-3.pdf} \\
%	\end{tabular}
%	\hspace{-7cm}
%	\caption{}\label{fig:2}
%\end{figure}

\bibliographystyle{elsarticle-harv}
\bibliography{biblio.bib}

\end{document}


